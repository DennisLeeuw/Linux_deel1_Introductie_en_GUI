De SuSE Linux distributie komt oorspronkelijk uit Duitsland en is een afgeleide van Slackware. De naam was in het begin een afkorting S.u.S.E. die stond voor: Software- und SystemEntwicklung. De eerste release was in 1994 en dat maakt SuSE tot een van de oudste commerci\"ele distributies. In 2003 is SuSE gekocht door Novell dat toen nog groot was in de server-markt. Novell is degene geweest die OpenSUSE, de niet commerci\"ele versie op de markt gezet heeft in 2005. Daarna is SuSE nog verscheidene malen overgenomen door ander bedrijven. Op dit moment is het in handen van EQT, een Europese investeringsmaatschappij.

SuSE is vooral bekend geworden om door de YaST, Yet Another Setup Tool. De interface waarmee de Linux distributie op een eenvoudige manier beheerd en ge\"installeerd kan worden.
