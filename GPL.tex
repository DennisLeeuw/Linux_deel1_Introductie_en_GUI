De GPL van de Free Software Foundation voegt er vele regels aan toe. De belangrijkste regel hierin is dat als je wat
wijzigt in de software dan moet je die gewijzigde broncode ook weer openbaar maken. Wat bij de vorige twee licenties
mogelijk is is de code te nemen, er wijzigingen in aan te brengen en dan alleen een gecompileerde (binaire) versie te
leveren zonder de broncode. Dat mag met de GPL niet meer. Elke wijziging die niet alleen voor je zelf maakt, daarvan
moet ook de broncode weer publiek beschikbaar zijn.\par

Als het goed is, althans dat is het doel van deze licentie, wordt elke verbetering zo openbaar en wordt de software
beter. Je kunt een probleem dat je opgelost hebt dus niet meer alleen voor jezelf houden.
