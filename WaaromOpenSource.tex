Wat Linux en het GNU project bijzonder maken is het feit dat alle code open source is. Je mag er mee doen en laten wat
je wil, je mag het aanpassen, je kan het inzien en je mag het gebruiken en voor het merendeel nog gratis ook. Dat is
natuurlijk bijzonder in een wereld die draait om commercie. Waarom is het dan toch zo populair? Om daar een goed
antwoord op te geven moeten we eerst eens kijken wat software nu eigenlijk is en hoe het werkt.\par

De eerst versies van Unix zoals geschreven door Ken Thompson, Dennis Ritchie en de overige leden
van het team bij Bell Labs was geschreven is assembly language. Assembly language is een taal die heel dicht ligt tegen
wat computers snappen en daarmee altijd systeemafhankelijk is. In de tijd dat Unix werd ontwikkeld geloofde men dat je
assembly language nodig had om een besturingssysteem snel genoeg te laten zijn. Dennis Ritchie nam de taal B ontwikkeld
door Ken Thompson, maakte verbeteringen en kwam met \index{C}C in
1972. In 1973 kwam Unix versie 4 uit die voor een groot deel herschreven was in C en daarmee aantoonde dat een hogere
programmeertaal gebruikt kon worden om besturingssystemen in te schrijven, maar belangrijker nog omdat er een hogere
programmeertaal werd gebruikt was Unix opeens portable naar andere machines.\par

Voor het programmeren in C heb je een C-compiler, een linker en een C-Library nodig. Library is het Engelse woord voor
bibliotheek een C-bibliotheek bevat een aantal standaardfuncties die je kan gebruiken in een programma, zo hoeft niet
elke programmeur de printf functie te programmeren. Functies kunnen hergebruikt worden door het aanroepen van de
library. Een heel simpel programma als Hello World ziet er in C zo uit:

\lstinputlisting[language=C]{helloworld.c}

De printf functie die de woorden ``Hello World!'' op het scherm laat zien is zo'n standaard functie uit de C-Library.\par

De compiler is verantwoordelijk voor het omzetten van de C-code in machinetaal, het compileren,
en de linker zorgt ervoor dat de functie uit de C-library gelinkt wordt aan het programma dat je geschreven hebt. En zo
kunnen we dus simpel stukjes programmatuur (functies) hergebruiken door er een library van te maken.\par

Er zijn twee manieren waarop de linker ervoor kan zorgen dat de printf-functie gelinked kan worden met je programma. Het
kan statisch en dynamisch. Statisch betekend dat een kopie van de functie toegevoegd wordt aan je programma. Je
programma wordt daarmee onafhankelijk van de C-library. Dynamisch linken betekent dat in je programma een verwijzing
komt te staan naar de printf-functie in die specifieke C-library. Je C-programma wordt zo afhankelijk van de specifieke
versie van de C-library die aanwezig is op je systeem. Dat is geen probleem zolang je het programma gebruikt op
systemen die dezelfde C-library hebben als jij, zoals het geval is bij mensen die dezelfde distributie gebruiken als
jij. Het voordeel is dat je eigen programma veel kleiner wordt en dus makkelijker te verspreiden is. Ook als er een
kleine wijziging gemaakt wordt in de printf-functie die geen invloed heeft op de binaire syntax van de printf-functie
dan kan je programma gelijk gebruik maken van deze verbetering doordat je alleen de C-library update. \ Je hoeft je
programma dan niet opnieuw te bouwen, compileren.\par

Daar zit het verschil in API en ABI compatibiliteit. De API van een functie is de syntax van de functie, als deze
verandert dan moet je je programma opnieuw compileren en als het tegenzit moet je zelfs je code aanpassen. ABI
compatibiliteit is veel complexer, maar kleine wijzigingen kunnen zoals gezegd soms doorgevoerd worden zonder dat de
applicatie er last van heeft.\par

Een programma dat alleen als binary, dus als compilde versie wordt uitgeleverd kan dus alleen draaien op systemen die
gelijk zijn aan het systeem waarop de code gecompiled is, zoals op Windows en Mac OS X systemen gebeurt. Ook de
applicaties op je telefoon zijn vaak afhankelijk van de juiste versie van het OS.\par

Als je echter de beschikking hebt over de broncode dan kan je die code ook compilen op je eigen computer en zorgen dat
die ook werkt op jou systeem. Je bent dan niet meer afhankelijk van een leverancier die jou systeem moet ondersteunen.
Soms moet je wel wat aanpassingen maken om het geheel goed te laten werken. De grafische interface van Windows is heel
anders dan die van Mac OS X, en daar zit dan ook een heel andere library onder. Dus als je een Windows applicatie op
een Mac wil compileren zul je wel wat programmeerwerk moeten doen. Maar als ik een applicatie heb die geschreven is
voor Debian kan ik hem meestal zonder enige wijziging compileren op CentOS.\par

En daar zit de kracht van open source door programma's, met het delen van de \ broncode wordt de reikwijdte van systemen
waarop de software kan draaien veel groter. Maar zoals gezegd moet je soms wel wijzigen maken om het te laten werken,
dus het is bijna een eis dat je de software mag aanpassen en dat is dan ook in veel open source licenties vastgelegd.
