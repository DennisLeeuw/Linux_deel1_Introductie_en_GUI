Het auteursrecht stamt uit 1710 en was oorspronkelijk bedoeld om de drukker te beschermen. Later is dit over gegaan op
de auteur. Omdat ook een programmeur een schrijver is geldt er voor broncode dezelfde rechten als op boeken. Maar wat
houdt auteursrecht nu eigenlijk in.

{\selectlanguage{dutch}
Volgens de wet:}

{\selectlanguage{dutch}
{\textquotedbl}Het auteursrecht is het uitsluitend recht van de maker van een werk van letterkunde,
wetenschap of kunst, of van diens rechtverkrijgenden, om dit openbaar te maken en te verveelvoudigen, behoudens de
beperkingen, bij de wet gesteld.{\textquotedbl} (Artikel 1 Auteurswet) }

{\selectlanguage{dutch}
Dit zegt dus iets over het openbaar maken en het kopi\"eren (verveelvoudigen) van het gemaakte werk. De auteur mag dit
doen en niemand anders (uitsluitend recht van de maker). Als een programmeur software schrift en dit aan iemand anders
geeft, dan mag deze de software gebruiken maar niet kopi\"eren en verder verspreiden.}

{\selectlanguage{dutch}
Voor de auteur zijn de regels dus in de wet vastgelegd, welke rechten een gebruiker van de software heeft dat mag de
auteur zelf bepalen. Deze rechten worden vastgelegd in een gebruikerslicentie. Elke auteur kan zijn eigen licentie
maken, wat veel bedrijven dan ook doen. De meest bekende is waarschijnlijk Microsoft's \index{EULA}EULA, End-Users
License Agreement. De EULA zegt in het kort dat Microsoft niet verantwoordelijk is voor de gemaakt software en
eventuele fouten die het mocht bevatten, je het op 1 machine mag gebruiken, je mag geen kopie\"en mag maken en het mag
maar door 1 persoon tegelijkertijd worden gebruikt.}

{\selectlanguage{dutch}
Omdat Unix ontstaan is uit een systeem van software met elkaar delen zijn er andere licenties ontstaan. De
meestvoorkomende zullen we in dit hoofdstuk behandelen.}

