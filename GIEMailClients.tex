Mozilla levert naast de browser Firefox ook een open source \index{e-mail!e-mail client}e-mail client met de naam
\index{Thunderbird}Thunderbird dit is een volwaardige e-mail client inclusief kalenderfunctionaliteit.

{\selectlanguage{dutch}
Een e-mail client die erg lijkt op Microsoft Outlook is \index{Evolution}Evolution. Sinds versie 2.8 is het onderdeel
van de GNOME project en Evolution is dan ook standaard ge\"installeerd op CentOS.}

{\selectlanguage{dutch}
Evolution en Thunderbird draaien ook op Windows en Mac OS X.}

{\selectlanguage{dutch}
Het KDE project heeft ook zijn eigen e-mail client en die heet \index{KMail}KMail. Standaard zijn er dus al vele e-mail
clients om uit te kiezen. Als je Op Internet gaat zoeken zijn er nog veel meer smaken beschikbaar. Dat is een van de
vele voordelen van open source, anderen zeggen een nadeel, er zijn enorm veel keuzes.}

{\selectlanguage{dutch}
[TODO] basis gebruik Evolution koppelen met google account (IMAP)}
