Jaren lang was het meest gebruikte en dominante office pakket dat van Microsoft. Het was
beschikbaar voor Windows en Mac OS, maar niet voor Linux systemen. Oorspronkelijk had een Duitse student, Marco
B\"orries, StarWriter ontwikkeld om zijn studie in te documenteren. Later richtte hij een bedrijf op genaamd Star
Division en werd het een office pakket met de naam StarOffice. Het bedrijf werd in 1999 opgekocht door Sun Microsystems
die het pakket open source maakte onder de naam OpenOffice.org.

{\selectlanguage{dutch}
\foreignlanguage{dutch}{In 2009 werd Sun Microsystems gekocht door Oracle. Er was veel twijfel over wat Oracle met
OpenOffice.org wilde en dat zorgde voor een fork, een kopie van de code, die bekend werd onder de naam LibreOffice in
2010 en beheerd wordt door The Document Foundation. Oracle bracht }\foreignlanguage{dutch}{uiteindelijk in 2011 de code
van OpenOffice.org onder bij de Apache Foundation, helaas zat toen het merendeel van de ontwikkelaars al bij
LibreOffice. Beide projecten bestaan nog steeds, maar de meeste distributies leveren LibreOffice mee.}}

{\selectlanguage{dutch}
\foreignlanguage{dutch}{LibreOffice gebruikt standaard het Open Document Format voor al zijn documenten en is gratis
beschikbaar. Voor Linux, Mac OS X en Windows kan je het pakket vanaf de website downloaden
}\href{https://www.libreoffice.org/}{https://www.libreoffice.org}\foreignlanguage{dutch}{, maar dat hoeven wij niet te
doen omdat we het al meege\"installeerd hebben tijdens de installatie van CentOS.}}

{\selectlanguage{dutch}
LibreOffice bevat de volgende software onderdelen}

\begin{itemize}
\item {\selectlanguage{dutch}
Writer -- Tekstverwerking}
\item {\selectlanguage{dutch}
Calc -- Spreadsheets}
\item {\selectlanguage{dutch}
Impress -- Presentaties}
\item {\selectlanguage{dutch}
Draw -- Tekenpakket}
\item {\selectlanguage{dutch}
Math -- Een formule editor}
\item {\selectlanguage{dutch}
Base -- Database}
\end{itemize}
{\selectlanguage{dutch}
LibreOffice maakt standaard gebruik van het Open Document Format. Een officieel erkent en gestandaardiseerd
bestandsformaat dat er voor zorgt dat data altijd weer te lezen is omdat exact beschreven is hoe een document opgebouwd
moet zijn. De belangrijkste bestandsformaten zijn:}

\begin{itemize}
\item {\selectlanguage{dutch}
odt -- Open Document Text}
\item {\selectlanguage{dutch}
ods -- Open Document Spreadsheet}
\item {\selectlanguage{dutch}
odp -- Open Document Presentation}
\item {\selectlanguage{dutch}
odi -- Open Document Image, bitmap format}
\item {\selectlanguage{dutch}
odg -- Open Document Graphic, vector format}
\item {\selectlanguage{dutch}
odf -- Open Document Formula}
\item {\selectlanguage{dutch}
odb -- Open Document Database}
\end{itemize}
