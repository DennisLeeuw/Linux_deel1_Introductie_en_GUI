Het Linux systeem is rijk aan multimedia applicaties. Omdat van ouds veel codecs, vertallers van analoog naar digitaal, voorzien waren van patenten zijn er ook veel open source en patent vrije codecs ontwikkeld. Voor audio zijn dat o.a. FLAC (Free and Lossless Audio Codec) en Ogg Vorbis dat een vervanger is voor bijvoorbeeld MP3. Voor video is er Theora ontwikkeld. Al deze standaarden zijn ondergebracht bij de Xiph Foundation, een non-profit organisatie die zich bezig houd met de ontwikkeling en het ondersteunen van het ontwikkelen van open standaarden op het gebied van multimedia.
