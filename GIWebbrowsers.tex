Een populaire browser is Mozilla \index{Firefox}Firefox. Het KDE-project heeft zijn eigen webbrowser. De KDE browser bestaat uit een engine
en en een interface. De engine is een library die alle benodigde functies voor het afhandelen van webpagina's heeft.
Het is ooit begonnen als KHTML. \ , de engine van deze browser,
\index{WebKit}WebKit wordt inmiddels ook door Apple gebruikt voor
zijn Safari browser.

Toen Google zijn eigen webbrowser ontwikkelde werd de basis hiervan vrij gegeven als open source
browser met de naam \index{Chromium}Chromium. Google gebruikt zelf
ook deze basis voor zijn Chrome browser, maar voegt daar nog wat eigen elementen aan toe. De open source versie is ook
op Linux te gebruiken en wordt door veel distributies meegeleverd en is makkelijk later te installeren.

[TODO] Basisgebruik firefox

