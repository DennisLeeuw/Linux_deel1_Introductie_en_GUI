Een van de allereerste distributies was het Softlanding Linux System (SLS) door Peter MacDonald in 1992, deze Linux distributie bevatte als eerste ook een grafische interface. Het stond bekend om zijn buggy character en er ontstonden dan ook al snel opvolgers zoals Slackware van Patrick Volkerding dat in 1993 uitkwam, de oudste nog steeds bestaande distributie, en Yggdrasil. Ook Debian is een afgeleide van SLS.

Een Linux distributie is een collectie van software samen met de Linux-kernel. Veel van de software is vaak afkomstig van het GNU-project. In dit hoofdstuk geven we een overzicht wat er zich zoal in een distributie kan bevinden. Dit betekend niet dat je alle genoemde software ook daadwerkelijk in elke distributie tegenkomt. Makers van een distributie maken hun eigen keuzes welke software zij belangrijk vinden, maar met dit hoofdstuk krijg je een beeld met wat er zoals mogelijk is.

In dit boek zullen we je laten kennis maken met twee dominante Linux distributies die van Red Hat en die van Debian.

