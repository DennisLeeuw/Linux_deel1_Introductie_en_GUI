Een verwarring die weleens wil ontstaan is het verschil tussen open source en open standaarden en toch is daar een
wezenlijk verschil.\par

Open standaarden beschrijven hoe bijvoorbeeld data uitgewisseld kan worden. Protocollen als SMTP, POP3, IMAP, Telnet en
FTP zijn allemaal beschreven in documenten die vrij op Internet toegankelijk zijn. Iedereen, dus ook de open source
wereld, kan deze standaarden implementeren en er dus voor zorgen dat verschillende systemen, open source en
commercieel, met elkaar kunnen communiceren. Gesloten protocollen die door een bedrijf zijn bedacht kunnen alleen door
dat bedrijf gebruikt worden, hoewel door luisteren op het netwerk er natuurlijk ook gekeken kan worden hoe het protocol
werkt.
